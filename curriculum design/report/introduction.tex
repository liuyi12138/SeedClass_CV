\section{引言}
随着科学研究和技术的不断推进,我们已经进入了一个信息爆炸的时代。而图像是人们从客观世界获取信息的重要来源,因为人类是通过感觉器官从客观世界获取信息的,即通过耳、目、口、鼻、手通过听、看、味、嗅和接触的方式获取信息,而在这些信息中视觉信息占据70\%\cite{zhy2007review}。因此数字图像处理在通信领域和计算机视觉领域占据了举足轻重的地位。

在大量的数字图像应用领域,人们经常期望得到高分辨率(简称HR)图像。但受限制于设备,传感器亦或者信道干扰等因素,我们得到的图像往往是低分辨率图像(LR)。增加空间分辨率最直接的解决方法就是通过传感器制造技术减少像素尺寸(例如增加每单元面积的像素数量);另外一个增加空间分辨率的方法是增加芯片的尺寸,从而增加图像的容量。因为很难提高大容量的偶合转换率,所以这种方法一般不认为是有效的,因此,引出了图像超分辨率重建技术。

超分辨率图像重建(Super resolution image re-construction, SRIR或SR)是指用信号处理和图像处理的方法,通过软件算法的方式将已有的低分辨率(Low-resolution, LR)图像转换成高分辨率(High-resolution, HR)图像的技术.它在视频监控(Video surveillance)、图像打印(Image printing)、刑侦分析(Criminal investigation analysis)、医学图像处理(Medical image processing)、卫星成像(Satellite imaging)等领域有较广泛的应用\cite{Su2013review}. 

因此本文以图像的超分辨率重建为研究目标,提出了一种基于Conditional Gated PixelCNN的方法实现针对低分辨率的人脸图像的超分辨率恢复。本文首先介绍了研究的背景,接着讨论了相关领域前人的工作,然后提出了我们基于Conditional Gated PixeCNN的算法设计及其改进,设计并展示了相应的对比实验和结果,最终我们将讨论我们的研究结果并总结我们的工作。