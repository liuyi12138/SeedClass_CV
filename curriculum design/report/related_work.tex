\section{相关工作}
在上世纪80∼90年代,就有人开始研究超分辨率图像重建的方法,1984年Tsai的论文\cite{tsai1984multiframe}是最早提出这个问题的文献之一. 1998年, Borman等\cite{borman1998spatial}发表了一篇超分辨率图像重建的综述文章. 2003年, IEEE Signal Processing Magazine发布了一期超分辨率图像重建的专刊\cite{park2003super}. 这些较早期的综述文章主要介绍传统的基于重建的超分辨率算法的研究情况.

而近年来,关于图像超分辨率的重建的研究逐渐出现明确的方向分化,其中针对单一图像的超分辨率重建的研究可以分为无训练样本的增强边缘的单帧超分辨率研究和有训练样本的基于学习的单帧超分辨率研究\cite{Su2013review}。无训练样本的增强边缘超分辨率重建本质上其实是对图像边缘信息的增强处理,主要通过传统图像插值技术。但是由于线性模型无法表达输入和输出之间的复杂依赖关系,因此这些方法由于缺乏可表达性而受到困扰。在实践中,此类方法通常无法充分预测高频细节,从而导致高分辨率输出模糊。基于学习的方法在单帧图像超分辨率重建领域里表现出色。它们采用机器学习方法从大量训练样本中提取图像的高频特征,并构建映射模型,从而对未知测试样本图像的信息进行预测,达到提高图像分辨率的目的。

CNN是很好一种提取图像信息的网络结构,董等人\cite{dong2015image}采用了以MSE损失为度量的三层CNN实现了低分辨率到高分辨率的初步映射,而Kim等人\cite{kim2016accurate}通过将深度提高到了20层并仅学习高分辨率图像和插值的低分辨率图像的残差,从而提高了精度。陈等人\cite{Chen2020Res}使用了具有更大感受野的深层残差网络,并引入局部残差学习和全局残差学习相结合的方法来提高学习率,与现有的Bicubic和SRCNN相比重建效果得到了提升。

此外,采用对抗损失才训练网络也是一个很好的超分辨率重建的研究方向。Metz等人\cite{metz2016unrolled}介绍了一种通过定义针对鉴别器的展开优化的生成器目标来稳定生成对抗网络的方法,从而实现有效的防止模式崩溃和数据多样性缺乏的问题。Ledig等人\cite{ledig2017photo}提出了用于图像超分辨率的网络SRGAN,使用感知相似性而非像素空间相似性引起的内容损失作为损失度量,实现了以4倍为因子的像素放大。

然而基于CNN的网络模型无法很好的修补超分辨率的图像细节,而基于GAN的网络模型常常面临无法覆盖到数据样本的多样性的问题。针对GAN的现存问题,我们寻找到了一种基于PixelCNN的先验概率生成模型\cite{oord2016pixel,van2016conditional},它能够有效地处理与给定低分辨率图像相关联的高分辨率图像的多样性。我们还将PixelCNN与CNN网络进行结合,从而有效解决了图像重建的细节修复问题。